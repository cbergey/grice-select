% Template for Cogsci submission with R Markdown

% Stuff changed from original Markdown PLOS Template
\documentclass[10pt, letterpaper]{article}

\usepackage{cogsci}
\usepackage{pslatex}
\usepackage{float}
\usepackage{caption}

% amsmath package, useful for mathematical formulas
\usepackage{amsmath}

% amssymb package, useful for mathematical symbols
\usepackage{amssymb}

% hyperref package, useful for hyperlinks
\usepackage{hyperref}

% graphicx package, useful for including eps and pdf graphics
% include graphics with the command \includegraphics
\usepackage{graphicx}

% Sweave(-like)
\usepackage{fancyvrb}
\DefineVerbatimEnvironment{Sinput}{Verbatim}{fontshape=sl}
\DefineVerbatimEnvironment{Soutput}{Verbatim}{}
\DefineVerbatimEnvironment{Scode}{Verbatim}{fontshape=sl}
\newenvironment{Schunk}{}{}
\DefineVerbatimEnvironment{Code}{Verbatim}{}
\DefineVerbatimEnvironment{CodeInput}{Verbatim}{fontshape=sl}
\DefineVerbatimEnvironment{CodeOutput}{Verbatim}{}
\newenvironment{CodeChunk}{}{}

% cite package, to clean up citations in the main text. Do not remove.
\usepackage{apacite}

% KM added 1/4/18 to allow control of blind submission


\usepackage{color}

% Use doublespacing - comment out for single spacing
%\usepackage{setspace}
%\doublespacing


% % Text layout
% \topmargin 0.0cm
% \oddsidemargin 0.5cm
% \evensidemargin 0.5cm
% \textwidth 16cm
% \textheight 21cm

\title{Listeners use descriptive contrast to disambiguate novel referents}



\begin{document}

\maketitle

\begin{abstract}
Human listeners are often faced with referential ambiguity. Description
is one cue listeners can use to narrow down referents. Beyond narrowing
potential referents to those that match a descriptor, listeners may
further infer that a described object is one that contrasts with other
relevant objects of the same type (e.g., ``The tall cup'' contrasts with
another, shorter cup). This kind of contrastive inference is known to
operate in online processing as listeners try to visually identify a
referent as an utterance progresses (Sedivy et al., 1999). However, it
is not known whether listeners use this type of inference to explicitly
guide their choice of a referent. In three experiments, we test whether
adult listeners use color and size adjectives contrastively to guide
their mapping of a novel word onto a novel referent. We find that while
participants use size adjectives contrastively to guide referent choice,
they do not do so using color adjectives (Experiment 1). Further, even
when color is described with more relative language (Experiment 2) or
emphasized with prosodic contrastive stress (Experiment 3), participants
do not consistently interpret color contrastively to guide referent
choice. These results demonstrate that listeners are able to use
adjective contrast to disambiguate a novel word's referent, but do not
treat all adjective types as equally contrastive.

\textbf{Keywords:}
reference resolution; pragmatics; prenominal adjectives
\end{abstract}

\section{Introduction}\label{introduction}

When trying to communicate, human listeners are faced with uncertainty.
Novice listeners---children---face a continuous speech stream filled
with unknown words referring to unformed concepts. Even seasoned
listeners---adults---contend with noise, variable pronunciation,
ambiguous meanings, and the occasional unknown word, too. Fortunately,
listeners bring sensitive phonetic, syntactic, and semantic skills to
the task, allowing them to reduce ambiguity during conversations and
over developmental time. Most of these well-documented skills are
concerned with the listener's understanding of the speaker's utterance
alone. But communication occurs in context: in a rich world to which
language refers. Listeners' ability to combine utterance information
with context---their pragmatic ability---may be a powerful tool in
resolving referential ambiguity.

One potential pragmatic tool for reducing referential uncertainty is
contrastive inference. Contrastive inferences are those inferences that
derive from the principle that description should discriminate. This
principle falls out of the more general Gricean maxim that speakers
should say as much as they need to say and no more (Grice, 1975). To the
extent that communicators strive to be minimal and informative,
description should discriminate between the referent and some relevant
contrasting set. This contrastive inference is fairly obvious from some
types of description, such as some postnominal modifiers: ``The door
with the lock'' clearly implies a contrasting door without one (Ni,
1996; Sedivy, 2002, 2003). The degree of contrast implied by more common
descriptive forms, such as prenominal adjectives in English, is less
clear. Speakers do not always use prenominal adjectives contrastively,
often describing more than is needed to establish reference (Engelhardt,
Bailey, \& Ferreira, 2006; Mangold \& Pobel, 1988; Pechmann, 1989). How,
then, do listeners interpret these descriptions?

Sedivy and colleagues carried out a visual world task demonstrating that
adults interpret at least some prenominal adjective use as contrastive
(Sedivy, K. Tanenhaus, Chambers, \& Carlson, 1999). In their task, four
objects appeared on a screen: a target (e.g., a tall cup), a contrastive
pair (e.g., a short cup), a competitor that shares the target's feature
but not category (e.g., a tall pitcher), and an irrelevant distractor.
Participants then heard a referential expression: ``Pick up the tall
cup.'' Adults looked more quickly to the correct object when the
utterance referred to an object with a same-category contrastive pair
(tall cup vs.~short cup) than when it referred to an object without a
contrastive pair (e.g., the tall pitcher). Their results suggest that
listeners expect speakers to use prenominal description when they are
distinguishing between potential referents of the same type, and
listeners use this inference to rapidly allocate their attention to the
target as an utterance progresses. This effect was demonstrated for size
and material adjectives; the results for color adjectives were mixed
(Sedivy, 2003; Sedivy et al., 1999). More recently, this contrastive
processing effect was replicated with 5-year-old participants using size
adjectives (Huang \& Snedeker, 2008). These experiments demonstrate that
listeners interpret at least some prenominal adjectives contrastively,
and use this contrastive inference to guide their attention allocation.
These results leave open, however, whether listeners use prenominal
adjective contrast to resolve referential ambiguity and explicitly guide
their referent choice.

In order to determine whether adults can use prenominal adjective
contrast to disambiguate referents, and how those inferences are
affected by adjective type, we use a reference game with novel objects.
Novel objects provide both a useful experimental tool and an especially
interesting testing ground for contrastive inferences. These objects
avoid effects of typicality and familiarity that relate to level of
description in production (Pechmann, 1989; Rubio-Fernández, 2016) on
particular features (Mangold \& Pobel, 1988). They have unknown names
and feature distributions, creating the ambiguity necessary for our test
of referential disambiguation. But the ability to disambiguate novel
referents, or to establish reference with incomplete information, is
also the broader problem of learning about the world. This skill would
aid not only adult speakers dealing with ambiguous or degraded
communicative signal, but also children who need to establish new
word--referent mappings. Across the developmental span, contrastive
inference could help listeners exploit regularities in language and
their environment to learn about both.

\section{Experiment 1}\label{experiment-1}

In Experiment 1, we test whether adult participants use prenominal
adjective contrast to choose a novel referent. To examine whether
contrast occurs across adjective types, we test participants in two
conditions: color contrast and size contrast. In a task similar to that
of Sedivy and colleagues (1999), we present participants with arrays of
novel fruit objects. On critical trials, participants see a target
object, a lure object that shares the target's contrast feature but not
its shape, and a contrastive pair that shares the target's shape but not
its contrast feature. Participants hear an utterance denoting the
feature: ``Find the {[}blue/big{]} dax.'' For the target object, use of
the adjective is necessary to disambiguate from the same-shape
distractor; for the lure, the adjective would be superfluous
description. If participants use contrastive inference to choose novel
referents, they should choose the target object. However, we do not
expect listeners to treat color and size equally. Because color is often
used redundantly in English while size is not (Nadig \& Sedivy, 2002;
Pechmann, 1989), we expect size to hold more contrastive weight,
encouraging a more consistent contrastive inference.

\begin{CodeChunk}
\begin{figure}[H]

{\centering \includegraphics{figs/colortrial-1} 

}

\caption[On the left]{On the left: an example of a contrastive trial in which the critical feature is size. Here, the participant would hear the instruction ``Find the small dax.'' On the right: an example of a contrastive trial in which the critical feature is color. Here, the participant would hear the instruction ``Find the red dax.'' In both cases, the target is the top object.}\label{fig:colortrial}
\end{figure}
\end{CodeChunk}

\subsection{Method}\label{method}

\subsubsection{Participants.}\label{participants.}

100 participants were recruited from Amazon Mechanical Turk. First, 20
participants were recruited to pilot the task with size adjectives.
After the pilot, a full run of the experiment was conducted: 40
participants were assigned to a condition in which the critical feature
was color (stimuli contrasted on color), and 40 participants were
assigned to a condition in which the critical feature was size.

\subsubsection{Stimuli.}\label{stimuli.}

Stimulus displays were arrays of three novel fruit objects. Fruits were
chosen randomly at each trial from 25 fruit kinds. Ten of the 25 fruit
drawings were adapted and redrawn from Kanwisher, Woods, Iacoboni, \&
Mazziotta (1997); we designed the remaining 15 fruit kinds. Each fruit
kind has an instance in each of four colors (red, blue, green, or
purple) and two sizes (big or small). There were two display types:
unique target displays and contrastive displays. Unique target displays
contain a target object that has a unique shape and is unique on the
trial's critical feature (color or size), and two distractor objects
that match each other's (but not the target's) shape and critical
feature. Contrastive displays contain a target, its contrastive pair
(matches the target's shape but not critical feature), and a lure
(matches the target's critical feature but not shape). The positions of
the target and distractor items were randomized within a triad
configuration.

\begin{CodeChunk}
\begin{figure*}[tb]

{\centering \includegraphics{figs/e1_fig-1} 

}

\caption[Proportion of times that participants chose the target and lure items as a function of condition and whether an adjective was provided]{Proportion of times that participants chose the target and lure items as a function of condition and whether an adjective was provided. Points indicate group means; error bars indicate 95\% confidence intervals computed by non-parametric bootstrapping.}\label{fig:e1_fig}
\end{figure*}
\end{CodeChunk}

\subsubsection{Design and Procedure.}\label{design-and-procedure.}

Participants were told they would play a game in which they would search
for strange alien fruits. In the size condition, each participant saw
eight trials; in the color condition, each participant saw twelve
trials. Half of the trials were unique target displays and half were
contrastive displays. Crossed with display type, half of trials had
audio instructions that described the critical feature of the target
(``Find the {[}blue/big{]} dax''), and half of trials had audio
instructions with no adjective description (``Find the dax''). A name
was randomly chosen at each trial from a list of twelve nonce names:
dax, blicket, wug, toma, gade, sprock, koba, zorp, flib, boti, quen, and
lomet. In the size condition, the size of the target (big or small) was
also crossed with display type and instruction type. All experiment code
is available on GitHub and will be accessible at time of unblinding.

\subsection{Results}\label{results}

We first confirmed that participants understood the task by analyzing
performance on trials in which there was a target unique on both shape
and the relevant adjective. The below results hold for the pilot with
size adjectives, but we report values from the full run of both
conditions. We asked whether participants chose the target more often
than expected by chance (\(33\%\)) by fitting a mixed effects logistic
regression with an intercept term, a random effect of subject, and an
offset of \(logit(1/3)\) to set chance probability to the correct level.
The intercept term was reliably different from zero for both color
(\(\beta =\) 1.13, \(t =\) 4.01, \(p\) \textless{} .001) and size
(\(\beta =\) 7.61, \(t =\) 3.95, \(p\) \textless{} .001). In addition,
participants were more likely to select the target when an adjective was
provided in the audio instruction in both conditions. We confirmed this
effect statistically by fitting a mixed effects logistic regression
predicting target selection from condition, adjective use, and their
interaction with random effects of participants. Adjective type (color
vs.~size) was not statistically related to target choice (\(\beta =\)
0.99, \(t =\) 1.61, \(p =\) .107), and adjective description in the
audio increased target choice (\(\beta =\) 1.93, \(t =\) 4.63, \(p\)
\textless{} .001). The two effects did not interact (\(\beta =\) -1.06,
\(t =\) -1.71, \(p =\) .087). Participants had a general tendency to
choose the target in unique target trials, which was amplified if the
audio instruction contained the relevant contrast adjective.

Our key test was whether participants would choose the target object on
contrastive trials in which description was given, reflecting use of a
contrastive inference to choose a novel referent. To do this, we compare
participants' rate of choosing the target to their rate of choosing the
lure, which shares the relevant contrast feature with the target, when
the audio described the contrast feature. In the pilot data with only
size adjectives, participants chose the target significantly more often
than they chose the lure, demonstrating a contrastive inference in their
referent choice (\(\beta =\) 1.03, \(t =\) 2.79, \(p =\) .005). In the
full run, participants chose the target more than the lure in the size
condition (\(\beta =\) 0.85, \(t =\) 1.97, \(p =\) .049), though more
marginally. However, participants in the color condition did not choose
the target significantly more often than they chose the lure
(\(\beta =\) 0.24, \(t =\) 1.23, \(p =\) .218). On contrastive trials in
which a descriptor was not given, participants dispreferred the target,
instead choosing the lure object, which matched the target on the
descriptor but had a unique shape; this was true across color
(\(\beta =\) -1.75, \(t =\) -2.02, \(p =\) .043) and size (\(\beta =\)
-7.96, \(t =\) -3.73, \(p =\) \textless{} .001) conditions. Adjective
use therefore increased target choice (\(\beta =\) 1.37, \(t =\) 4.59,
\(p\) \textless{} .001) across contrastive trials. Participants' choice
of the target in the size condition was therefore not due to a prior
preference for the target in contrastive displays, but relied on
contrastive interpretation of the adjective.

\section{Experiment 2}\label{experiment-2}

\begin{CodeChunk}
\begin{figure*}[tb]

{\centering \includegraphics{figs/e2_fig-1} 

}

\caption[Experiment 2 referent choice on contrastive display trials, by adjective type and whether the adjective was specified in the audio instruction]{Experiment 2 referent choice on contrastive display trials, by adjective type and whether the adjective was specified in the audio instruction.}\label{fig:e2_fig}
\end{figure*}
\end{CodeChunk}

The results of Experiment 1 demonstrate that adult listeners interpret
color and size adjectives differently, attributing more contrastive
weight to size adjectives and using them to choose novel referents
accordingly. Why might adult listeners do this? As alluded to earlier,
participants' responses may reflect a symmetry between adjective
production and comprehension. Since color adjectives are used more
frequently and redundantly than size adjectives, listeners may attenuate
the contrastive weight of color description. A second possibility, not
mutually exclusive with the first, is that size is inherently more
contrastive than color because size is scalar and relative whereas color
is more discrete. Listeners may interpret size as more contrastive
because it occurs on a continuum or because it requires comparison to
other category exemplars. In our second experiment, we aim to examine
whether manipulating the descriptor---making it more or less scalar---of
the same stimulus will change participants' contrastive inferences about
color. To do so, we present stimuli with varying levels of saturation.
These stimuli can be contrasted on either discrete color (blue, grey),
relative color (bluer, greyer), or relative brightness (bright, dark)
adjectives. By maintaining a constant set of stimuli, we can rule out
explanations of the contrastive differences between adjective types that
are stimulus-specific---such as some contrasts being more visually
salient---to which our first experiment is susceptible.

\begin{CodeChunk}
\begin{figure}[H]

{\centering \includegraphics{figs/brightdarktrial-1} 

}

\caption[Example of a contrastive trial in which the critical feature is saturation]{Example of a contrastive trial in which the critical feature is saturation. Here, the participant would hear the instruction ``Find the [red/redder/bright] dax.'' The target is the top object. The lure is on the bottom left.}\label{fig:brightdarktrial}
\end{figure}
\end{CodeChunk}

\subsection{Methods}\label{methods}

\subsubsection{Participants.}\label{participants.-1}

One hundred and twenty participants were recruited from Amazon
Mechanical Turk. Forty participants were assigned to a condition in
which the critical contrast was discrete color (blue, grey), 40
participants were assigned to a condition in which the critical feature
was relative color (bluer, greyer), and 40 participants were assigned to
a condition in which the critical feature was relative brightness
(bright, dark).

\subsubsection{Stimuli \& Procedure.}\label{stimuli-procedure.}

Stimulus displays were arrays of three novel fruit objects, similar to
those used in Experiment 1. Each fruit kind had an instance in each of
four colors (red, blue, green, or purple) in two saturation levels
(saturated or unsaturated). As in Experiment 1, there were two display
types: unique object displays and contrastive displays. All fruit
objects on any given display were the same hue. In a unique object
display, the target differs from the two distractors in shape and in
saturation. In a contrastive display, the target matches one distractor
in shape but not in saturation, and the other distractor (the lure) in
saturation but not in shape. Instructions and trial structure were
identical to Experiment 1. Audio instructions diverged, this time
reflecting the critical contrasts of each condition: discrete color
(red, blue, green, or purple; or grey), relative color (redder, bluer,
greener, or purpler; or greyer), and relative brightness (bright or
dark).

\subsection{Results}\label{results-1}

We first analyzed performance on trials in which there was a target
unique on both shape and the relevant contrast adjective, asking whether
participants chose the target more often than expected by chance
(\(33\%\)) by fitting a mixed effects logistic regression with an
intercept term, and a random effect of subject, and an offset of
\(logit(1/3)\) to set chance probability to the correct level. The
intercept term was reliably different from zero for color (e.g., blue,
grey) (\(\beta =\) 4.02, \(t =\) 2.75, \(p =\) .006), relative color
(e.g., bluer, greyer) (\(\beta =\) 2.74, \(t =\) 5.31, \(p\) \textless{}
.001), and brightness (e.g., bright, dark) (\(\beta =\) 2.37, \(t =\)
5.19, \(p\) \textless{} .001). Adjective use did not significantly
increase target choice (\(\beta =\) -0.37, \(t =\) -1.05, \(p =\) .294).
Participants had a general tendency to choose the target in unique
target trials, but there was no significant main effect of adjective use
on these trials.

Our key test was whether participants would choose the target object on
contrastive trials when description was given, reflecting use of a
contrastive inference to choose a novel referent. Across all adjective
types, participants did not choose the target significantly more than
they chose the lure, in fact showing a numerical preference for the lure
in each condition and a significant preference for the lure in the
discrete color (e.g., blue, grey) condition (for discrete color:
\(\beta =\) -1.64, \(t =\) -3.34, \(p =\) .001; for relative color:
\(\beta =\) -0.94, \(t =\) -1.84, \(p =\) .066; for brightness:
\(\beta =\) -0.1, \(t =\) -0.44, \(p =\) .659). Preference for the lure
was also demonstrated when the adjective was not specified (for discrete
color: \(\beta =\) -7.2, \(t =\) -3.71, \(p =\) \textless{} .001; for
relative color: \(\beta =\) -9.8, \(t =\) -3.32, \(p =\) .001; for
brightness: \(\beta =\) -0.99, \(t =\) -2.57, \(p =\) .010). Across
these conditions, participants did not consistently demonstrate a
contrastive interpretation in their referent choices, and in fact tended
to avoid contrastive choices. This may in part be due to a strong prior
preference for the lure object, but nonetheless means that contrastive
selection did not emerge.

\section{Experiment 3}\label{experiment-3}

\begin{CodeChunk}
\begin{figure*}[tb]

{\centering \includegraphics{figs/e3_fig-1} 

}

\caption[Experiment 3 referent choice on contrastive display trials, by adjective type (color or size) and whether the adjective was specified (with prosodic stress) in the audio instruction]{Experiment 3 referent choice on contrastive display trials, by adjective type (color or size) and whether the adjective was specified (with prosodic stress) in the audio instruction.}\label{fig:e3_fig}
\end{figure*}
\end{CodeChunk}

The results of Experiment 2 suggest that color is not easily interpreted
contrastively, even when descriptors suggest a scalar interpretation.
Given that color has contrastive weight in implicit measures in specific
contexts (Sedivy, 2003; Sedivy et al., 1999), and fails to do so in our
explicit measure, perhaps participants needed a more explicit pragmatic
cue to contrastiveness. To test whether an explicit pragmatic cue would
induce contrastive inferences, in Experiment 3 we manipulate prosody.

The placement of prosodic stress on a word tends to evoke its set of
alternatives, inducing a contrastive interpretation. In early work on
incremental semantic processing (Eberhard, Spivey-Knowlton, Sedivy, \&
Tanenhaus, 1995), participants saw displays containing, for example, a
large blue square, a small blue square, a large yellow circle, and a
small red triangle. Hearing an instruction with contrastive stress on
the size adjective---``Find the \emph{large} blue square''---facilitated
their fixation on the target compared to instructions without
contrastive stress. In Experiment 3, we test whether adding contrastive
stress on the adjective in our instructions facilitates an explicit
contrastive referent choice.

\subsection{Methods}\label{methods-1}

\subsubsection{Participants.}\label{participants.-2}

Eighty participants were recruited from Amazon Mechanical Turk. Forty
participants were assigned to a condition identical to the color
contrast condition in Experiment 1, with the modification that audio
stimuli included stress on the color adjective. Forty participants were
assigned to a condition identical to the size contrast condition in
Experiment 1, with the modification that audio stimuli included stress
on the size adjective.

\subsubsection{Stimuli and Procedure.}\label{stimuli-and-procedure.}

Stimulus displays were identical to those in Experiment 1. Audio stimuli
were recorded such that utterances containing an adjective had
contrastive stress on the adjective (e.g., ``Find the \emph{small}
blicket''). Instructions and trial structure were identical to
Experiment 1.

\subsection{Results}\label{results-2}

We first asked whether participants chose the target more often than
expected by chance (\(33\%\)) on unique target trials by fitting a mixed
effects logistic regression with an intercept term, and a random effect
of subject, and an offset of \(logit(1/3)\) to set chance probability to
the correct level. The intercept term was reliably different from zero
for the color (\(\beta =\) 2.15, \(t =\) 6.25, \(p\) \textless{} .001)
and size (\(\beta =\) 2.58, \(t =\) 4.35, \(p\) \textless{} .001)
conditions. Adjective type (color vs.~size) was not statistically
related to target choice (\(\beta =\) 0.42, \(t =\) 0.77, \(p =\) .444),
and adjective use did not significantly increase target choice
(\(\beta =\) 0.34, \(t =\) 0.92, \(p =\) .360).

Our key test was whether participants would choose the target object on
contrastive trials when description was given with contrastive stress,
reflecting use of the descriptor and/or prosodic stress to choose the
referent. In the size condition, participants chose the target more
often than they chose the lure, though this difference did not reach
significance (\(\beta =\) 1.37, \(t =\) 1.44, \(p =\) .151). In the
color condition, participants chose the lure numerically more than they
chose the target, and the difference was not significant (\(\beta =\)
-0.51, \(t =\) -1.09, \(p =\) .277). In this task, contrastive stress
did not encourage a contrastive referent choice. Though the data
patterned similarly to that of Experiment 1, with participants in the
size condition choosing the target more often than they chose the lure,
this difference did not reach significance.

\section{Discussion}\label{discussion}

In this series of experiments, we asked whether listeners could use
pragmatic contrast to resolve referential ambiguity. Participants were
able to use size adjectives contrastively to establish a novel
word--referent mapping. Their contrastive inference goes beyond the
implicit attention allocation shown in prior eye-tracking paradigms
(Huang \& Snedeker, 2008; Sedivy et al., 1999), determining explicit
referent choice. This finding bolsters contrastive inference as a viable
tool for referential disambiguation.

Participants failed, however, to use color adjectives contrastively in
choosing referents. What makes size different from color? One
possibility is that the scalar nature of size supports a contrastive
interpretation. We tested whether using relative color adjectives (e.g.,
bluer, greyer) or adjectives describing value (bright, dark) on
saturated and desaturated stimuli would encourage the contrastive
inference. We also tested whether adding a prosodic cue to contrast
(e.g., ``Find the \emph{blue} dax'') would encourage contrastive
inference. Participants persisted in interpreting color
non-contrastively, never consistently choosing the intended target over
the lure. Though we do not claim that contrastive color inferences
cannot be used to explicitly choose referents, it seems that a
contrastive interpretation is difficult to elicit using color, while it
emerges under similar conditions using size.

Another possibility is that color adjectives are often used redundantly,
and therefore receive less contrastive weight than adjectives
consistently used to differentiate between referents. Sedivy (2003) puts
forth such an account, finding that color adjectives tend not to be
interpreted contrastively in eye-tracking measures except in contexts
that make their use unlikely. In comparison, adjectives describing
material (e.g., plastic) and size are interpreted contrastively, which
corresponds to less redundant use of material and size adjectives in
production (see Chapter 10 of Gibson \& Pearlmutter, 2011; Sedivy,
2003). This account explains well why color is not interpreted
contrastively here, but fails to explain why presumably rare adjectives
(bluer, bright) do not receive contrastive treatment in our task.
Further work is necessary to determine whether contrastive inferences
hew to production norms, and whether implicit indications of contrast
usually extend to explicit referent choice.

Though the participants in our experiments were adults, the ability to
disambiguate novel referents using contrast most obviously serves
budding language learners: children. Contrastive use of adjectives is a
pragmatic regularity in language that children could potentially exploit
to establish word--referent mappings. Tasks using a mixture of novel
adjectives and words suggest that children as young as 3 can make
contrastive inferences about adjectives (Diesendruck, Hall, \& Graham,
2006; Gelman \& Markman, 1985; Huang \& Snedeker, 2008). We plan to
research further the development of these contrastive skills, as well as
their potential as tools for extracting information from language and
context.

\section{Conclusion}\label{conclusion}

We establish here that adult listeners are able to use contrastive
inference to map novel words to novel referents. This ability is
limited, however: it emerges with size but not color description. This
result accords with findings that size adjectives more reliably evoke
contrast in eye-tracking measures (Sedivy, 2003). Our manipulations in
Experiment 2 to make color more relative, which did not result in
contrastive inference, suggest that an explanation of size's effect
based only on its scalar nature is insufficient. An account that relates
production norms to listener interpretation may better explain our
findings. Further research to determine the relationship between
contrastive production and contrastive inference across adjective types,
as well as the relationship between implicit measures of contrastive
inference and explicit referent choice, is ongoing.

\section*{References}\label{references}
\addcontentsline{toc}{section}{References}

\hypertarget{refs}{}
\hypertarget{ref-diesendruck_childrens_2006}{}
Diesendruck, G., Hall, D. G., \& Graham, S. A. (2006). Children's Use of
Syntactic and Pragmatic Knowledge in the Interpretation of Novel
Adjectives. \emph{Child Development}, \emph{77}(1), 16--30.

\hypertarget{ref-eberhard_eye_1995}{}
Eberhard, K. M., Spivey-Knowlton, M. J., Sedivy, J. C., \& Tanenhaus, M.
K. (1995). Eye movements as a window into real-time spoken language
comprehension in natural contexts. \emph{Journal of Psycholinguistic
Research}, \emph{24}(6), 409--436.

\hypertarget{ref-engelhardt_speakers_2006}{}
Engelhardt, P. E., Bailey, K. G. D., \& Ferreira, F. (2006). Do speakers
and listeners observe the Gricean Maxim of Quantity? \emph{Journal of
Memory and Language}, \emph{54}(4), 554--573.

\hypertarget{ref-gelman_implicit_1985}{}
Gelman, S. A., \& Markman, E. M. (1985). Implicit contrast in adjectives
vs. nouns: Implications for word-learning in preschoolers*.
\emph{Journal of Child Language}, \emph{12}(1), 125--143.

\hypertarget{ref-gibson_processing_2011}{}
Gibson, E. A., \& Pearlmutter, N. J. (2011). \emph{The Processing and
Acquisition of Reference}. MIT Press.

\hypertarget{ref-grice1975logic}{}
Grice, H. P. (1975). Logic and conversation. \emph{1975}, 41--58.

\hypertarget{ref-huangsnedeker2008}{}
Huang, Y. T., \& Snedeker, J. (2008). Use of referential context in
children's language processing. \emph{Proceedings of the 30th Annual
Meeting of the Cognitive Science Society}.

\hypertarget{ref-kanwisher}{}
Kanwisher, N., Woods, R. P., Iacoboni, M., \& Mazziotta, J. C. (1997). A
locus in human extrastriate cortex for visual shape analysis.
\emph{Journal of Cognitive Neuroscience}, \emph{9}(1), 133--142.

\hypertarget{ref-mangold_informativeness_1988}{}
Mangold, R., \& Pobel, R. (1988). Informativeness and Instrumentality in
Referential Communication. \emph{Journal of Language and Social
Psychology}, \emph{7}(3-4), 181--191.

\hypertarget{ref-nadig_evidence_2002}{}
Nadig, A. S., \& Sedivy, J. C. (2002). Evidence of Perspective-Taking
Constraints in Children's On-Line Reference Resolution.
\emph{Psychological Science}, \emph{13}(4), 329--336.

\hypertarget{ref-nietal}{}
Ni, W. (1996). Sidestepping garden paths: Assessing the contributions of
syntax, semantics and plausibility in resolving ambiguities.
\emph{Language and Cognitive Processes}, \emph{11}(3), 283--334.

\hypertarget{ref-pechmann_incremental_1989}{}
Pechmann, T. (1989). Incremental speech production and referential
overspecification. \emph{Linguistics}, \emph{27}(1), 89--110.

\hypertarget{ref-rubio-fernandez_how_2016}{}
Rubio-Fernández, P. (2016). How Redundant Are Redundant Color
Adjectives? An Efficiency-Based Analysis of Color Overspecification.
\emph{Frontiers in Psychology}, \emph{7}.

\hypertarget{ref-sedivy_invoking_2002}{}
Sedivy, J. C. (2002). Invoking Discourse-Based Contrast Sets and
Resolving Syntactic Ambiguities. \emph{Journal of Memory and Language},
\emph{46}(2), 341--370.

\hypertarget{ref-sedivy_pragmatic_2003-2}{}
Sedivy, J. C. (2003). Pragmatic Versus Form-Based Accounts of
Referential Contrast: Evidence for Effects of Informativity
Expectations. \emph{Journal of Psycholinguistic Research}, \emph{32}(1),
3--23.

\hypertarget{ref-sedivy_achieving_1999}{}
Sedivy, J. C., K. Tanenhaus, M., Chambers, C. G., \& Carlson, G. N.
(1999). Achieving incremental semantic interpretation through contextual
representation. \emph{Cognition}, \emph{71}(2), 109--147.

\bibliographystyle{apacite}


\end{document}
